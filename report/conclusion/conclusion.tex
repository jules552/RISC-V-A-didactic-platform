Overall I'm really proud of what I've made during this semester on the project and don't regret at all going for this project.
I've learned and relearned a lot of things about computer architecture that will for sure consolidate my knowledge for the future
since I think I'll be working in this field. 
I've also learned a lot about the RISC-V architecture and how to implement it in Verilog and I think it is a really good architecture 
to learn about computer architecture since it is really simple and easy to understand.
I still think RISC-V is at the beginning of its life and will be a really good alternative to ARM in the future since it is open-source
but some of the tools to develop on it are still not as good as the other architectures and I've encountered some documentation issue 
during the project and finding some information was sometimes harder than it should have been.
I loved the autonomy and freedom that Professor Kluter gave me during the project and it corresponded more to my way of working
than some other projects, I've done during my studies. I think the time I spent on it was around the time I'd planned so I'm quite happy
that the project didn't take the time I should have dedicated to other courses.\\

What should be improved in the future?
I think the project is a really good start for a RISC-V processor but I think adding some of the standard extensions could 
be a great start for any student that would like to continue the project. Like said in the testing section, I think some testbenches 
can be improved and some more can be added to test the processor more thoroughly since I can't be 100\% sure that the processor 
is bug-free but most of the tests I've done manually have passed so I think it is a good start.\\

Thank you again to Professor Kluter for allowing me to work on this project and for some of the resources he gave me 
to help me during the project. I hope this project will be useful for future students and that it will be improved in the future.
I hope you enjoyed reading this report and that it will help you understand the project better.

Of course the project is accessible on my GitHub page at \href{https://github.com/jules552/RISC-V-A-didactic-platform}{https://github.com/jules552/RISC-V-A-didactic-platform}